En este informe se aborda el diseño, modelación e implementación de un sistema de control para un sistema híbrido conformado por una celda de combustible y un banco de baterías, con el objetivo de regular el voltaje de una carga eléctrica a un valor de referencia de 250 V. La celda de combustible se caracteriza por una respuesta dinámica lenta pero capaz de entregar alta potencia, mientras que el sistema de baterías presenta una dinámica rápida pero con capacidad limitada para operación continua.

Para garantizar un control efectivo, se utiliza una estructura de lazos anidados que permite coordinar el aporte de ambos sistemas, de modo que en estado estacionario la celda de combustible asuma toda la carga y la corriente entregada por las baterías disminuya gradualmente a cero. Se propone el diseño de controladores mediante el método del lugar de la raíz, considerando las limitaciones físicas de los actuadores y empleando técnicas para evitar la saturación y el fenómeno de wind-up.

El trabajo contempla tanto el desarrollo teórico del sistema de control, como su modelación y simulación en Matlab/Simulink, donde se evalúa el desempeño del sistema ante diversas perturbaciones en la carga, con y sin mecanismos anti wind-up y prealimentación. De esta manera, se busca validar la eficacia de la propuesta y analizar el comportamiento dinámico del sistema bajo diferentes condiciones operativas.