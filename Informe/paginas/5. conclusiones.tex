\section{Conclusiones}

A partir del desarrollo realizado, se concluye que el sistema de control diseñado permite regular de manera efectiva el 
voltaje de la carga a 250 [V], cumpliendo con el objetivo planteado. La estrategia de lazos anidados 
utilizada, junto con el diseño de controladores PI por método del lugar de las raíces, permitió una coordinación 
adecuada entre la celda de combustible y el banco de baterías, considerando las diferentes dinámicas de cada subsistema.

En las simulaciones se observó que el sistema responde correctamente frente a perturbaciones en la carga. La batería actúa 
de forma inmediata ante cambios bruscos gracias a su mayor ancho de banda, mientras que la celda de combustible 
asume gradualmente la demanda, tendiendo a suministrar la totalidad de la corriente en estado estacionario.

Además, se evidenció la importancia de implementar estrategias como el anti-windup, ya que su ausencia produce 
saturaciones prolongadas, sobrepasos elevados y pérdida del control. Por otro lado, la inclusión del término de prealimentación 
(feedforward) mejora la respuesta dinámica del sistema, reduciendo los sobrepasos y tiempos en zona de saturación, 
lo que permite una actuación más eficiente de los controladores.

Finalmente, la modelación en Simulink permitió observar el comportamiento del sistema bajo distintos escenarios y validar la eficacia de 
los controladores implementados. El sistema mostró un comportamiento coherente con el análisis teórico, y se destaca la robustez 
del control frente a cambios abruptos en la resistencia de carga.

\newpage